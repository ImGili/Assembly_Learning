\documentclass{elegantnote}
\title{汇编笔记}
\author{顾磊欣}

\begin{document}
    \maketitle
    \section{基础知识}
    \subsection*{监测点1.1}
    (1)一个CPU的寻址能力为8KB,那么它的地址总线的宽度为\underline{13}。\par
    解答:
    因为一个Bytes就是一个地址的大小,所以不需要再乘8了。
    \begin{align*}
        8\text{KB}&= 8\times2^{10}\text{Byte} \\
        & =2^3\times2^{10}\text{Byte} \\
        & = 2^{(3+10)}\text{Byte} \\
        & = 2^{13}\text{Byte}
    \end{align*}\par
    (2)1KB的储存器有\underline{$2^{10}$}个存储单元。存储单元的编号从\underline{0}到\underline{1023}。\par
    (3)1KB的存储器可以存储\underline{$2^{10}\times8$}个bit,\underline{$2^{10}$}个Byte。\par
    (4)1GB、1MB、1KB分别是\underline{$2^{10}\times2^{10}\times2^{10}$}、\underline{$2^{10}\times2^{10}$}、\underline{$2^{10}$}Byte。\par
    (5)8080、8088、80286、80386的地址总线宽度
    分别为16根、20根、24根、32根。则它们的寻址能力分别为:
    \underline{$\frac{2^{16}}{2^{10}}=2^6=64$}(KB)、
    \underline{$\frac{2^{20}}{2^{10}\times2^{10}}=1$}(MB)、
    \underline{$\frac{2^{24}}{2^{10}\times2^{10}}=2^4=16$}(MB)、
    \underline{$\frac{2^{32}}{2^{10}\times2^{10}\times2^{10}}=2^2=4$}(GB)。\par
    (6)8080、8088、8086、80286、80386的数据总线宽度
    分别为8根、8根、16根、16根、32根。则它们一次可以传送的数据为:
    \underline{$\frac{8}{8}=1$}(B)、\underline{$\frac{8}{8}=1$}(B)、\underline{$\frac{16}{8}=2$}(B)、\underline{$\frac{16}{8}=2$}(B)、\underline{$\frac{32}{8}=4$}(B)。\par
    (7)从内存中读取1024字节的数据,8086至少要读\underline{$\frac{1024}{2}=512$}次,80386至少要读\underline{$\frac{1024}{4}=256$}次。\par
    (8)在存储器中,数据和程序以\underline{二进制}形式存放。\par
    \newpage
    \section{寄存器}
    \subsection*{检测点2.1}
    (1)写出每条汇编指令执行后相关寄存器中的值。\par
    \begin{tabular}{lr}
        mov ax, 62627   &   AX=\underline{F4A3H} \\
        mov ah, 31H     &   AX=\underline{31A3H} \\
        mov al, 23H     &   AX=\underline{3123H} \\
        add ax, ax      &   AX=\underline{6246H} \\
        mov bx, 826CH   &   BX=\underline{826CH} \\
        mov cx, ax      &   CX=\underline{6246H} \\
        mov ax, bx      &   AX=\underline{826CH} \\
        add ax, bx      &   AX=\underline{04D8H} \\
        mov al, bh      &   AX=\underline{0482H} \\
        mov ah, bl      &   AX=\underline{6C82H} \\
        add ah, ah      &   AX=\underline{D882H} \\
        add al, 6       &   AX=\underline{D888H} \\
        add al, al      &   AX=\underline{D810H} \\
        mov ax, cx      &   AX=\underline{6246H} \\
    \end{tabular}\par
    (2)只能使用目前学过的汇编指令,最多使用4条指令,编程计算2的4次方。
    \begin{verbatim}
        mov ax, 2
        add ax, ax
        add ax, ax
        add ax, ax
    \end{verbatim}
\end{document}